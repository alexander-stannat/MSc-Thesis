\chapter{Conclusion and Discussion}
\label{chap:Conclusion and Discussion}

\noindent{}In this thesis we have examined reputation mechanisms in distributed systems and their resistance to different types of malicious behaviour, whereby we placed the largest emphasis on sybil attacks. We began by introducing a mathematical framework for our research in which we defined fundamental and pertinent concepts such as transaction sequences, work graphs, accounting mechanisms and allocation policies. Thereafter we mathematised different types of malicious behaviour, accounting mechanisms were supposed to prevent, i.e. lazy-freeriding, misreport attacks and sybil attacks. It was our goal to introduce requirements for accounting mechanisms to be resistant to this type of malicious behaviour. \vspace{1em}\\	


\noindent{}We began by investigating \textbf{lazy-freeriding} in chapter \ref{chap:Mathematical Framework for Accounting Mechanisms}. We discovered a combination of requirements that would ensure an accounting mechanism together with an allocation policy could successfully prevent lazy-freeriding. The requirement was called positive-report responsiveness, while the allocation policy had to satisfy the additional constraint of banning any nodes from the choice set with accounting values that exceeded a given lower bound. This resulted in agents, who contributed far fewer resources than they consumed, not being served anymore data by honest agents. Therefore it became impossible for agents to leech excessively.\vspace{1em}\\ 

\noindent{}Next we analysed \textbf{misreports} and the resistance of accounting mechanisms to these types of attacks in the network, whereby we began by critically examining the DropEdge protocol introduced by Seuken \& Parkes \cite{Sybil-proof Accounting Mechanisms with Transitive Trust}. We discovered that this mechanism was only resistant to particular types of misreports which were quite narrowly defined. In response to this discovery we expanded our definition of misreport-proofness and examined the TrustChain data structure in combination with generic gossip protocol. We concluded that TrustChain satisfied a stronger requirement for misreport-proofness for accounting mechanisms, provided that accounting mechanisms satisfied the property of positive-report responsiveness. Given this misreport-proofness we moved on to the most critical issue accounting mechanisms faced, namely sybil attacks. \vspace{1em}\\


\noindent{}The largest emphasis was placed on \textbf{sybil attacks} in this thesis. After having solved the issue of misreports by either the DropEdge mechanism or the TrustChain architecture we moved on to characterising the effects of sybil attacks on P2P networks. We began by defining the cost incurred by the attacker, given by the amount of work that had to be performed for the network and formalised the profit of an attack as well, whereby the profit of a sybil attack was given by the additional amount of work the attacker could consume after the sybil attack had been carried out. The fact that neither profit nor cost of sybil attacks have been rigorously defined up until this point proves itself to be quite problematic and we highlighted the necessity for these definitions. In attempting to determine the profit of such an attack, we realised that we had to determine the expected value of an infinite discrete stochastic process. To solve this problem we postulated an interaction model for nodes in the network. \vspace{1em}\\ 


\subsubsection*{Interaction Model}
\label{subsubsec:Interaction Model}
\noindent{}In order to determine the profit of a sybil attack, we introduced an interaction model by which participants transact with one another and compared the outcome of this interaction model with the real-world {\it Tribler} application. Within the construction of this model we analysed a number of different allocation policies for their resistance to sybil attacks and decided that the winner-takes-all policy was the most suitable in this endeavour. With the now won definiton of sybil attack profit, we realised that computing the value of the expected profit was practically impossible. This prompted us to reformulate the values of cost and profit in terms of the accounting values that sybil attackers were able to obtain. The advantage to this was that the profit in terms of accounting values was, in fact, computable and we were therefore able to gauge the effectiveness of a sybil attack. We did, however incur a problem with this definition. \vspace{1em}\\



\subsubsection*{Representativeness}
\label{subsubsec:Representativeness}
\noindent{}The values of sybil attack cost and profit in terms of accounting values were now much easier to compute. However, they were not actually the relevant metric, but just a proxy for the earlier defined cost and profit of sybil attacks in terms of work. Consequently, we were interested in determining the relationship between the two and came up with two example cases in which the sybil attack benefit converges to infinity in terms of one, but not the other. We learned that the two were not equivalent. This lead to the new definitions of weak and strong {\it representativeness}. After having introduced representativeness, we concluded that any accounting mechanism that is resistant to strongly beneficial sybil attacks in terms of accounting values and is weakly representative, is already resistant to strongly beneficial sybil attacks in terms of work, which was the desired property. \vspace{1em}\\



\subsubsection*{Impossibility Results}
\label{subsubsec:Impossibility Results}
\noindent{}Next, we analysed some existing results in the literature which stated that accounting mechnanisms satisfying independence of disconnected agents, symmetry and single-report responsiveness were \textbf{not} resistant to strongly beneficial sybil attacks in terms of work. We concluded that given our definitions of sybil attack cost and profit this result was incorrect. We corrected the result by adding the requirement of parallel-report responsiveness and added that the consequent sybil susceptibility was in terms of accounting values and not work. If an accounting mechanism satisfied the additional property of strong representativeness then the given sybil susceptibility was also in terms of work. We then extended the model to multiple hops ensuring that the upper assertions were true for arbitrary work graphs. We then introduced another property called serial-report responsiveness and made an analogous assertion. \vspace{1em}\\



\subsubsection*{Sybil-proofing Accounting Mechanisms}
\label{subsubsec:Sybil-proofing Accounting Mechanisms}
\noindent{}Using the impossibility results we arrived at in chapter \ref{chap:On the Impossibility of Sybil-Proofness}, we realised we could characterise passive sybil attacks as parallel and serial attacks and a combination of the two, which we called pyramid attacks. We inverted the concepts of parallel- and serial-report responsiveness to obtain resistance against these types of attacks. Thereafter we introduced the very important definition of multiple-path response bound from which we concluded that the profit in terms of accounting values of any sybil attack could be bounded by the profit of some pyramid attack, multiplied by a constant. This lead to the final result that any accounting mechanism satisfying the upper properties was resistant against strongly beneficial sybil attacks in terms of accounting values, which in combination with weak representativeness implied resistance to strongly beneficial sybil attacks in terms of work.\vspace{1em}\\ 

\noindent{}Given these results we believe that we have adequately answered the research question from chapter \ref{chap:Introduction} \vspace{1em}\\

\begin{center}
{\it What requirements does an accounting mechanism need to satisfy in order to effectively incentivise cooperation and prevent lazy freeriding, while being resistant to misreport attacks and mitigating the effects of sybil attacks?} \vspace{1em}\\
\end{center}

\section{Future Work}
\label{sec:Future Work}
\noindent{}In this thesis we covered a wide array of problems in the context of cooperation in P2P file sharing networks. We are particularly pleased with some of our results on the theoretical properties accounting mechanisms and allocation policies must satisfy in order to achieve sybil-proofness. Given the time constraints of a master thesis project, some of our research was cut short a bit. In this section we would like to elaborate a little bit on possible future work that could be conducted in researching the problems this thesis covered. \vspace{1em}\\

\subsubsection*{Allocation Policies}
\label{subsubsec:Allocation Policies}
\noindent{}The research on allocation policies we conducted within defining the sybil attack profit in terms of work, has been rather slim and  did not reach a final strong conclusion. We determined that out of the allocation policies introduced in chapter \ref{chap:Mathematical Framework for Accounting Mechanisms} the winner-takes all was the most sybil resistant and lead to the fairest distribution of data in the network as was seen in the experiments conducted in \ref{subsec:Experimental Evaluation}. In future work, one may want to formalise a more generic set of sensible allocation policies and determine the optimal policy out of this set. One may determine a set of allocation policies given by convex combinations of the top $n$ policies, i.e. combinations of distribution and top $n$ policies where the highest ranking $n$ nodes will be served each with an amount of work corresponding to their standing in the choice set. If combined with some banning element, whereby only nodes with accounting values greater than a given upper bound are served, such allocation policies may succeed in preventing both lazy freeriding and sybil attacks. Research on allocation policies to our knowledge has been rather scarce and may be a topic of research that hides promising results.\vspace{1em}\\

\subsubsection*{Resistance Against Weakly Beneficial Sybil Attacks}
\label{subsubsec:Resistance Against Weakly Beneficial Sybil Attacks}
\noindent{}The research conducted for this thesis was all done with P2P file sharing networks in mind, more precisely the {\it Tribler} application. In this setting weakly beneficial sybil attacks are not by any stretch of the imagination disastrous for the network. The reason for this is that in filesharing networks a weakly beneficial sybil attack in terms of work requires the attacker to invest infinite resources in order obtain infinite resources. This renders weakly beneficial sybil attacks comparatively harmless. No single malicious agent can simply demand all, or even a significant proportion of the resources in the network. However, other types of P2P networks such as social networks may not have such resistant properties. If one thinks of a sybil attacker on Facebook that aims to spread fake news by tricking the network into considering their content more relevant for people's feeds than it actually is. In such a case, a weakly beneficial sybil attack may have disastrous consequences. For future work it may be very nice to obtain some stricter finite upper bounds on the benefits of sybil attacks. For this, one could start of by tightening the definitions of parallel- and serial-report convergence to obtain a limit $\leq{}c$ for some $c>0$. \vspace{1em}\\

\subsubsection*{Expanding on Representativeness of Accounting Mechanisms}
\label{subsubsec:Expanding on Representativeness of Accounting Mechanisms}
\noindent{}In chapter \ref{chap:Representativeness of Accounting Mechanisms} we introduced the concept of representativeness of accounting mechanisms. The idea was that accounting values were simply a representation of the reputation of nodes in the network and therefore a proxy for the amount of work these nodes were entitled to. We incurred the problem of sybil attacks which were strongly beneficial in terms of accounting values but not in terms of work, and vice versa, prompting us to make the restriction for accounting mechanisms to have to satisfy at least weak representativeness in order to be sensible. In remark \ref{rem:Representativeness Function} to further elaborate on what requirements accounting mechanisms must satisfy in order to be weakly and strongly representative we explained the concept of a representativeness function, but did not delve further into this issue. It is, however a crucial point in the resistance to sybil attacks. In future work one may want to further identify properties representative accounting mechanisms satisfy and make consequent additional restrictions to be able to better identify sensible accounting mechanisms.\vspace{1em}\\ 

\section{Further Discussion and Ethical Ramifications}
\label{sec:Further Discussion and Ethical Ramifications}
\noindent{}The research question of this thesis has far-reaching implications and may find application in many types of networks other than just P2P filesharing networks. Digital currencies may be one of these, where the accounting values would not necessarily reflect an agent's trustworthiness, but instead the balance of their account, i.e. the amount of digital money they own. Another preeminent setting for accounting mechanisms to find application in, are online social networks. Networks such as Facebook and Twitter struggle to clamp down on the spreading of hateful speech and fake news. Malicious agents may decide to create many fake accounts which all "follow" or "like" one another in order to boost the probability of their content being seen by many people. The algorithms of these companies that decide whose content is recommended and shown on other honest agents' feeds so far have been very susceptible to these types of attacks. Online social networks try to shut down attacks like this with the help of machine learning algorithms that are trained to detect fake accounts. However, oftentimes mistakes are made and anyone who has used these services has witnessed this first hand. Companies running these services may want to consider broadening their set of tools with which they tackle these attacks by sybil-proof accounting mechanisms that represent the trustworthiness of agents in the network. \vspace{1em}\\

\noindent{}One more application that has occured to us throughout the process of this research have been real-life social scoring systems. Countries like China have introduced social credit systems with which they aim to rate their citizens trustworthiness. Bad behaviour of citizens will lead to lower scores while good behaviour will increase respective scores \cite{Big Data Meets Big Brother as China Moves to Rate its Citizens}. These scores are then used by the government to allow their citizens different levels of liberty, whereby lower ranking citizens are restricted in their freedom. From a western libertarian perspective this extent of government surveillance seems obviously unethical (of course this is debatable). We are aware of the fact that our research in the direction of social reputation scores may assist oppresive regimes in constructing social accounting mechanisms to enhance their control over peoples' lives. Of course, we realise that our research is only very peripherally related, nevertheless we feel that this must be pointed out and further research in this direction should be conducted with some level of caution and awareness of its ethical consequences.\vspace{1em}\\

\begin{comment}
\begin{itemize}
\item Future Research: Different Allocation Policies: Our research on allocation policies has been mostly inconclusive, figure out which allocation policy is most sybil resistant; We only covered \textbf{strongly} beneficial sybil attacks because of the P2P filesharing system in mind; In other types of networks this may not be sufficient at all, think about introducing an upper bound for sybil attack profit under certain conditions; Maybe mention that pur definitions of sybil attack cost and profit are based on our personal intuition, i.e. they just kinda make sense to us ... lol. Say that under other definitions of these, the whole thing may look very different.

\item Applications: P2P filesharing, obviously; Other types of P2P networks such as currency systems maybe; Social networks (big problems there with fake news propagation and fake account stuff); Social scoring systems ... tongue-in-cheek (this does make us think)

\item Ethical Ramifications of our research; An important thing here is that it could be theoreticall used for social accounting, malicous governments may use these things to track and alayse people. This is fucked up!
\end{itemize}
\end{comment}